\documentclass{article}
\usepackage[utf8]{inputenc}
\usepackage[margin=1in]{geometry}

\newcommand{\glossaryentry}[2]{#1 - #2}

\setlength{\parskip}{1em}
\setlength{\parindent}{0em}

\title{A Comparison of Immutable and Mutable Data Structures}
\author{}
\date{}

\begin{document}

\maketitle

\begin{center}
By:  Chi Jon, Daniil Furmanov, and Justin Harrison

\hfill

Submitted to Jordan Berard,

In partial fulfillment of the requirements of ENL1813T

\hfill

Algonquin College,

Computer Engineering Technology

\hfill

April 1, 2016
\end{center}

\section{Glossary}
\glossaryentry{Data Structure}{
  A format to efficiently store information. A common data
  structure is the array, where data is stored sequentially in memory.
}

\glossaryentry{Immutable Data Structure}{
  A data structure that cannot be modified in-place.
  Instead, a new copy with the modifications is created.
}

\glossaryentry{Mutable Data Structure}{
  A data structure that can be modified in-place.
}

\section{Introduction}
The purpose of this comparison is to determine which type of data structure,
immutable or mutable, is best for a range of programming problems.

\section{Performance}

\section{Ease of Use}

\section{Safety}

\end{document}
